\documentclass{article}

\usepackage{amsmath}
\usepackage{amsfonts}
\usepackage{hyperref}

\title{Elliptic curve arithmetic}
\author{Hendrik}

\begin{document}
\maketitle

\section{Arithmetic over the real numbers}

Consider the elliptic curve 
\begin{equation} \label{eq:elliptic_curve} y^2 =
    x^3 + 7 
\end{equation} 
over the real numbers, and denote a point $g$ by its projective coordinates
$(x_g, y_g, 1) \sim (\alpha x, \alpha y, \alpha)$. We also write the point
reflected over the $x$-axis as $-g = (x_g, -y_g, 1)$. In this projective space,
the line of infinities is given by $(x, y, 0)$ from which we further denote
$\infty = (0, 1, 0)$.

Given the set of points that are on the elliptic curve plus $\infty$, we
introduce a group structure as follows. For each $g$ and $h$ we let $g + h = r$
if $g$, $h$ and $-r$ are colinear. This definition implies that $g + \infty =
\infty + g = g$, as well that that $g + (-g) = \infty$. Hence, $e \equiv \infty$
is the unit element whereas $-g$ is the inverse of $g$. Lastly, the
sum $g + g$ is taken to be the inverse of the point that is colinear with the
tangent to the curve at $g$.  It is easily verified that this group operation is
commutative, since the colinearity is independent of the order of it's
generating points. The operation is also associative, although this is not a
trivial observation. 

For any positive integer $n$ we can then define multiplication on the elliptic
curve by $n \cdot g = g + \cdots g$ where summation was repeated $n - 1$ times.

Note that the geometric definition given above ensures that the group is
well-defined in the sense that it is closed under addition. In preperation of
extending this definition to elliptic curves over closed fields, we first
derive the algebraic group operations for elliptic curves over the real numbers.

Addition of inverse elements are equal to $e$ by definition. We then
consider two points $g(x_g, y_g)$ and $h(x_h, y_h)$ on the elliptic
curve~\eqref{eq:elliptic_curve}, which are not eachother's inverses. These two
points define a straight line
\begin{equation}
    \label{eq:straight_pq}
    y = m (x - x_g) + y_g
\end{equation}
where $m = (y_g - y_h) / (x_g - x_h)$ if $g \neq h$, and $m = 3 x_g^2 / 2 y_g$
otherwise. The sum of $g$ and $h$ is then the point $-r$ where $r$ is the
intersection of~\eqref{eq:elliptic_curve} and~\eqref{eq:straight_pq}. Solving
for $x_r$ is done by consecutively factoring out $(x - x_g)$ and $(x - x_h)$.
Not suprisingly, the solutions for different and coincident points is the same
if expressed in terms of the gradient $m$, namely, $x_r = m^2 - x_g - x_h$. We
thus find that
\begin{equation}
\begin{split}
    g(x_g, y_g) + h(x_h, y_h) 
    &= (g + h)(
        m^2 - x_g - x_h, 
        -m (m^2 - 2x_g - x_h) - y_g)\\
    &= (g + h)(
        m^2 - x_g - x_h, 
        -m (m^2 - x_g - 2x_h) - y_h).
\end{split}
\end{equation}
These equations make the commutativity of the operation manifest.


\section{Arithmetic over a finite field}

\subsection{Group structure}

In this section we consider the elliptic curve~\eqref{eq:elliptic_curve} over
the finite field $\mathbb{F}_p$ where $p$ is a prime number. The field
$\mathbb{F}_p$ is the set of integers modulo $p$, and since $p$ is taken prime,
this field forms a group with respect to addition and multiplication. We thus
consider all points $g(x_g, y_g) \in \mathbb{F}_p^2$ that satisfy
\begin{equation} \label{eq:elliptic_curve_field} 
    y^2 \equiv x^3 + 7 \mod p.
\end{equation} 

Equivalently to what we did in the previous section, we introduce a group
structure over the points on~\eqref{eq:elliptic_curve_field} by adding a unit
element $e$ and defining $g + e = e + g = g$, $g + (-g) = e$ and
\begin{equation}
\begin{split}
    g + h = (&m^2 - x_g - x_h \mod p, \\
             &-m (m^2 - 2x_g - x_h) - y_g \mod p)
\end{split}
\end{equation}
where $m = (y_g - y_h)(x_g - x_h)^{-1} \mod p$ if $g \neq h$, and $m = 3 x_g^2 (2
y_g)^{-1} \mod p$ otherwise.

The order of the group is the number of elements in the group, which can be
calculated using Schoof's algorithm in polynomial (in the logarithm of) time.
It is worth noting that not every value $y_g^2$
in~\eqref{eq:elliptic_curve_field} will have a square root in $\mathbb{F}_p$,
whereas there are two in case it exists.

\subsection{Scalar multiplication}

Similarly to what we did in the context of real numbers, scalar multiplication
can be defined as $n \cdot g = g + \cdots g$. This computation can be sped up
significantly by considering the binary represention $n = \sum_k b_k 2^k$ such
that
\begin{equation}
    n \cdot g = \sum_k b_k (2^k \cdot g).
\end{equation}
Since each term in this sum requires a doubling of the point obtained in the
previous term, we have at most two times $\log_2 n$ additions to compute, which
is a huge gain compared to the $n$ additions in the brute-force calculation.

Let us then consider the elements generated by $g$, i.e., the set $\hat{g} = \{i
\cdot g\}_{i = 0 \ldots n - 1}$, where $n$ is the minimum number for which $n
\cdot g = e$, indeed implying that $n$ is the number of elements in $\hat{g}$.
Note that $\hat{g}$ is a subgroup on the elliptic curve, since $i \cdot g + j
\cdot g = (i + j) \cdot g$ is also an element of $\hat{g}$, and we also have
that $i \cdot g + (n - i) \cdot g = e$. It can be shown that for any commutative
multiplicative group, the order of $\hat{g}$ divides the order of the
supergroup.\footnote{ This follows from Lagrange's theorem. Indeed, consider all
right coset spaces $\hat{g} + h$ on the elliptic curve. Each coset has the same
order as $\hat{g}$, because they are all isomorphic to it. Furthermore, the
cosets are disjoint. This is so because if a point were in two cosets, all
points would be related by the left action of $\hat{g}$, so that the two cosets
must be the same. Therefore, the right cosets form a partition and have the same
order, from which it follows that the order of $\hat{g}$ divides the order of
the group on the elliptic curve. The division thus equals the number of right
cosets of $\hat{g}$.} Hence, if the elliptic curve has $N$ points, every point
generates a subgroup of order $n$ where $n$ divides $N$. This also means that
all generated subgroups are finite and cyclic. In particular, if the order of
the elliptic curve is prime, the two subgroups are the trivial one (generated by
the unit element) and the whole group itself.

In cryptography we will look for subgroups with prime order $n$, in which case
we are sure that the corresponding subgroup does not have proper subgroups
itself. Given such a large prime number all points $n' \cdot g$, where $n' =
N/n$ and for any $g$ on the elliptic curve, are generators of the considered
subgroup. This is so because $N \cdot g = 0$ as any subgroup order divides $N$.

\subsection{Discrete logarithm}

Given that we have a point $n \cdot g$ and the generating point $g$, the
discrete logarithm of $n\cdot g$ is the number $n$. Currently, the fastest way
to compute this is brute-force, i.e., to repeatedly add the inverse of $g$ until
you get $g$, which is $O(n)$. Therefore, the time needed to invert scalar
multiplication is exponentially longer than the multiplication itself.

In a cryptographic setting, we are considering a finite field for a very large
prime number. For example, in \verb|secp256k1| the elliptic
curve~\eqref{eq:elliptic_curve_field} is considered over $\mathbb{F}_p$ where\footnote{
    \url{http://www.secg.org/sec2-v2.pdf}} 
\begin{equation*}
    p = 2^{256} - 2^{32} - 2^9 - 2^8 - 2^7 - 2^6 - 2^4 - 2^0.
\end{equation*}
Given a private key $n$ of the order of $2^{256}$ and a generating point $g$, a
public key is quickly calculated from $n \cdot g$.  Reverting this would take a
huge amount of time, namely, of the order of $2^{248}$ more slowly than the key
generation itself. It is this discrepancy in calculation time which is the backbone
of the security of private-public key pairings.

\section{Digital signatures}

In this section we will discuss the Elliptic Curve Digital Signature Algorithm
(ECDSA), which is a digital signature protocol used for signing transactions in
various cryptocurrencies. First let us observe that a digital signature is a
mathematical scheme that ensures the authenticity of digital messages.\footnote{
    \url{https://en.wikipedia.org/wiki/Digital_signature}}
The authenticity is derived from the following three crucial properties of a
digital signature.
\begin{description}
    \item[Authentication] Any recepient of the messages that knows the private
        key of the sender, can verify the signature of the message.
    \item[Non-repudiation] Any agent that sees a message can verify the
        signature of the message. Note that this property implies
        authentication.
    \item[Integrity] A message that is signed cannot be altered without
        rendering the signature invalid.
\end{description}

Non-repudiation thus means that the signature can be verified without having
access to the private key. As anyone can verify the signature, the latter cannot
be denied by the authorizer.

Let $m$ be the transaction or message and $(k, K)$ a private-public key
pair, i.e., $K = k \cdot G$ for a generator $G$. To sign the message, an
ephemeral private key $t$ is considered. According to ECDSA, the
signature is then given by
\begin{equation}
    \label{eq:ecdsa_sig}
    s(m, k, r) \equiv t^{-1} \mathrm{SHA}(m) + rk \mod p,
\end{equation}
where $(r, \_) = t \cdot G$ and $\mathrm{SHA}$ is a secure hash algorithm,
e.g., $\mathrm{keccak256}$.

Given the digital signature~$s$, the private key $r$ and the public key $K$ of
the authorizer, anyone can check the signature by verifying whether
\begin{equation}
    (r, \_) = u_1 \cdot G + u_2 \cdot K, 
\end{equation}
where
\begin{equation}
    u_1 \equiv \mathrm{SHA}(m) s^{-1} \mod p \quad\mathrm{and}\quad
    u_2 \equiv r s^{-1} \mod p.
\end{equation}
Indeed, the verification will only work if $K$ is the public key corresponding
to $k$, which can be easily checked by substituting for the relevant equations.




\end{document}
